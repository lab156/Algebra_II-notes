\noindent\textbf{Algebra II Assignment 4 \hspace{\fill} Luis Berlioz}
\problem{V.1.19}
First we review that at most $mn$ terms will be needed to generate $K(u,v)$.
Any element in $a_j\in K(u)$ is of the form:
$$a_j = \sum_{i=0}^{m-1} b_{ij} u^i$$
Where $m=[K(u):K]$ and $b_{ij}\in K$. 
Additionally, each $c\in K(u,v)$ can be written in the form:
$$c= \sum_{j=0}^{n-1} a_{j} v^j$$
This, because $K(u,v)= K(u)(v)$. By expanding $a_j$ we get that any $c\in K(u,v)$:
$$c= \sum_{j=0}^{n-1} \sum_{i=0}^{m-1} b_{ij} u^i v^j$$
This means that no more than these terms are needed to generate the field $K(u,v)$.

We will prove the second part by means of the counterpositive: 
$$[K(u,v):K]<mn \implies (m,n)>1$$ 
Note that the following couple of equations hold:
\begin{gather*}
m[K(u,v):K(u)] = [K(u,v):K]\\
n[K(u,v):K(v)] = [K(u,v):K]
\end{gather*}
Then:
$$[K(u,v):K] = \frac{m\, n}{\gcd(m,n)}p$$
Our original assumption was that $[K(u,v):K]<mn$ so we conclude $\gcd(m,n)>1$. 

\problem{V.1.20}
\subsection*{Part (a)}
Since:
\begin{gather} 
    [ML:M][M:K] = [LM:K]\label{eq.1}\\
    [ML:L][L:K] = [LM:K]\label{eq.2}
\end{gather}
And by Theorem V.1.2 we see that $[LM:K]$ is finite if and only if both $[ML:L]$ and $[L:K]$ are.
The same applies to $[ML:M]$ and $[M:K]$. 

\subsection*{Part (b)}
Using equations (\ref{eq.1}) and (\ref{eq.2}) we get that $[M:K]| [LM:K]$ and $[L:K]| [LM:K]$. 

To check the second proposition, i.e. that $[LM:K]\leq [L:K][M:K]$. Note that if $\ell \in LM = L(M)$ then the dimension of $LM$ over $L$ is at most that of $LM$ over $K$, thus:
$$\ell = \sum_{k=1}^{[M:K]}\alpha_k m_k$$
Where the  $m_k\in M$ and the $\alpha_k \in L$ which have dimension $[L:K]$ over $K$ and thus the whole thing can be expressed as:
$$\ell =  \sum_{k=1}^{[M:K]}\sum_{j=1}^{[L:K]}\kappa_{j,k}m_kn_j$$
Where the $\kappa_{j,k}\in K$ and the $n_j\in L$. 
Note that although we may need to use different elements $m_kn_j$ the upper limit of both sums will always remain the same (in total at most $[M:K][L:K]$ terms). 
Therefore, the dimension of $[LM:K]$ is at most $[L:K][M:K]$. 
\subsection*{Part (c)}
The proof goes the same way a in Problem V.1.19 above.
In which we can take $L=K(u)$ and $M=K(v)$ because the hypothesis of the fields being finitely generated is unnecessary.
And it was not used in the proof above.

\subsection*{Part (d)}
If $L$ es algebraic over $K$ then $[L:K]$ is finite. 
Same with $[M:K]$, then:
$$[LM:K]\leq [L:K][M:K]$$
Thus, it is finite, which implies that any element in $LM$ has finite order.
Therefore it is finite.

\problem{V.1.21}
\subsection*{Part (a)}
We will prove this proposition by contradiction. Assume that $\ell \in (L\cap M)\backslash K$, then $[K(\ell):K]>1$. And simultaneusly assume that:
$$[LM:K] = [L:K][M:K]$$
Breaking up trough the intermediate field produces:
$$[LM:K(\ell)][K(\ell):K] = [L:K(\ell)][M:K(\ell)][K(\ell):K]^2$$
Cancelling the common factor to conclude more easily later:
$$[LM:K(\ell)] = [L:K(\ell)][M:K(\ell)][K(\ell):K]$$
Under the assumption that $[K(\ell):K]>1$ then it has to be the case that:
$$[LM:K(\ell)] > [L:K(\ell)][M:K(\ell)]$$
Which is impossible as it implies that something in $LM$ came out of nowhere (Banach-Tarski? -- Not in my shift). Also this was proven to never be the case in V.1.20 above. Therefore, we have reached a contradiction, and such an $\ell$ is impossible.

\subsection*{Part (b)}
Let $[L:K]=2$. Then $\{1,\ell\}$ is a basis for any $\ell\in L\backslash K$.
If additionally, $L\cap M = K$ and $p \in M\backslash L$ then a basis for $K(p)$ is $\{1,p,\ldots,p^{n-1}\}$ where the order of $p$ is bounded above by $[M:K]$. 

Note that $\ell\neq p$ and this means that a basis for $K(\ell,p)$ necessarily has $2n$ elements. The condition of the order of $L$ being specifically 2 enters here; without it, we could not guarante for instance, that $\ell^2\notin M$. 

In conclusion, we have:
$$[K(\ell, p):K] = 2[K(p):K]$$
We can add more elements and the equation is still true, thus:
$$[ML:K] = 2[M:K]$$

\subsection*{Part (c)}
Let us call $r=\sqrt[3] 2$ and
$$s=\sqrt[3] 2 \left( -\frac 12 + i \frac {\sqrt 3}2\right)$$
Both this roots have order 3 with basis $\{1,r,r^2\}$ and $\{1,s,s^2\}$ respectively.

It is the case that every element in $\QQ(s)$ is either rational or has nonzero imaginary part. This is because (and this is true about both fields) $\sqrt[3] 2$ and $\sqrt[3] 4$ are linearly independent over $\QQ$. This is the reason why $\QQ(r)\cap \QQ(s) =\QQ$. 

On the other hand, $\QQ(s,r)$ is still of order 3 over $\QQ$ since they are roots of $x^3-1$.

\problem{V.2.1}
\subsection*{Part (a)}
If $\sigma \neq 0$; let $\sigma(f) = \sigma(g)$ for $f,g\in F$. Then:
\begin{gather*}
    \sigma(f)- \sigma(g)=0\\
    \sigma(f-g)=0
\end{gather*}
Since nonzero ring homomorphisms cannot send units to zero, and all nonzero elements of a field are units, the kernel of $\sigma$ has to be only 0. Thus, $f-g=0$.

Also, the nonzero elements of $F^\times$ form a group under multiplication. Assuming again that $\sigma\neq 0$, then $\sigma(1_F) =1_F$ because the identity is the unique element with the behaviour $\sigma(1_F)\cdot x = x \sigma(1_F) = x$.


\subsection*{Part (b)}
$\aut F$ is a group since for any $\sigma,\tau,\phi\in F$:
\begin{description}
    \item[well-defined] 
\end{description}

\subsection*{Part (c)}
Note that $\aut_K F$ contains the \textbf{identity}, because the identity fixes everything. Also, it is \textbf{closed} under composition because  $\forall k\in K$ and $\sigma,\tau\in \aut_K F$ then $\tau(k)\in K$ whence $\sigma(\tau(k))\in K$. 

Lastly, if $\sigma\in \aut_KF$ then $\sigma^{-1}$ is also in $\aut_K F$. To see this, take any $k\in K$ then $k= \sigma(k)$.  And this implies that 
$$\sigma^{-1}(k) = \sigma^{-1}(\sigma(k)) = k$$
Therefore $\aut_K F$ is closed under taking the \textbf{inverse}.
    

\problem{V.2.2}
We will follow closely the hint given in the textbook. Let $\sigma\in \aut_\QQ \RR$. Note that $\sigma(x^2) = \sigma(x)^2$, and this implies that $\sigma$ sends nonnegative numbers to nonnegative numbers. 

This means $\sigma$ preserves order because if we have $x-y\geq 0$ i.e. $x-y$ is nonnegative; then $\sigma(x) - \sigma(y)\geq 0$.

Finally, for each real $x$ assume without loss of generality that $\sigma(x)\leq x$. Take $\{x_n \from n\in \NN\}$ to be the sequence such that $x_n\leq x$ and $x_n\to x$ (namely, a Dedekind cut for $x$, $sup\{x_n\} =x$ ). Thus for all $n\in \NN$:
$$\sigma(x_n)\leq \sigma(x)\leq x$$
Taking the limit and by the ``squeeze'' Theorem $\sigma(x) = x$. Therefore, anything in $\aut_\QQ \RR$ is the identity map.

\problem{V.2.3}
Let $\sigma \in \aut_\QQ\QQ(\sqrt d)$. Observe that:
$$d= \sigma(\sqrt d^2)= \sigma(\sqrt d)^2 $$
This means that $\sigma(\sqrt d)$ can be either equal to $\sqrt d$ or $-\sqrt d$. 

Since $\QQ(\sqrt d)= \spann\{ 1,\sqrt{d}\}$, taking $\sigma(\sqrt{d}) =\sqrt{d}$ gives the identity. And, taking the only other option $\sigma(\sqrt{d}) =-\sqrt{d}$ gives us the conjugate map. The conjugate map has order to 2 as we check:
$$\sigma\sigma(a+ b\sqrt{d}) = \sigma(a-b\sqrt{d}) = a+b\sqrt{d}$$
Therefore there are only two options for $\aut_\QQ \QQ(\sqrt{d})$ it is either the identity or a map of order 2; this has to be isomorphic to $\ZZ_2$.


\problem{V.2.4}
\underline{Claim:} The Galois group of $\QQ(\sqrt 2, \sqrt 3, \sqrt 5)$ is isomorphic to $\ZZ_2\oplus \ZZ_2 \oplus \ZZ_2$.

Observe that   $\QQ(\sqrt 2, \sqrt 3, \sqrt 5)=\QQ(\sqrt 2)( \sqrt 3) (\sqrt 5)$. Since the elements to which we are extending the rational field are all roots of prime numbers, the minimal polynomial is going to be irreducible at each succesive step. Specifically, $x^2-5$ is irreducible in $\QQ(\sqrt 2)( \sqrt 3)$
\problem{V.2.5}
\subsection*{Part (a)}
To show that $\QQ(\sqrt{d})$ is Galois over $\QQ$, we check it is indeed the case for the \emph{only} element different from the identity $\aut_\QQ \QQ(\sqrt{d})$. We have seen that this is the conjugate $\sigma(a+b\sqrt{d})=a-b\sqrt{d}$ then:
$$a+b\sqrt{d}=a-b\sqrt{d} \implies b=0$$

\subsection*{Part (b)}
Interestingly, the same proof given above (Problem V.2.3) for the case $\QQ(\sqrt{d})$ where $0<d\in \QQ$ works for $d= -1$ in $\RR$ just by replacing the symbols. Thus, the same conclusion applies, i.e. $\aut_\RR\RR(\sqrt{d} )\cong \ZZ_2$. Moreover, Part (a) works identically also.
