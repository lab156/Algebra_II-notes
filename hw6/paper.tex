\problem{V.3.9}
Assuming that $F$ is the algebraic closure of $K$, let $E$ be any algebraic field extension.
Let us call $\bar E$ the algebraic closure of $E$, which we know to exist by Theorem V.3.6.
Also by  Theorem V.3.6, we know that $\bar E \cong F$, and say $\sigma$ is the $K-$isomorphism between these two fields.
Therefore the restriction $\sigma|_E$ is $K-$monomorphism that we want

Conversely, let $f\in K[x]$. We will show that $f$ has a root in $F$.
Let $E$ be an extension over $F$ such that $f$ splits in $E$; and $u\in E$ a root of $f$.
We are assuming that there exists a $K-$monomorphism $\sigma\from E\to F$.
Therefore $\sigma(f(u))=f(\sigma(u))=0$, and $\sigma(u)\in F$ is a root of $f$. 

\problem{V.3.11}
\subsection*{both part (a) and (b)}
Let $T\subset K[x]$ be the set of all the irreducible polynomials of all $u\in F$. That is:
$$T=\{ f\in K[x] \from f \text{ is the irreducible polynomial of some } u\in F\}$$
Also, take $\bar K$ to be a splitting field of $T$ over $K$. 

There is no point in making a distinction between the finite and the infinitely generated cases. This is because we would unavoidably use the same method in proving both scenarios; just that in different levels of generality.

Now we can assert that $\bar K$ is a splitting field of a set of separable polynomials $T$.
Thus by theorem V.3.11 ((iii)$\implies $(ii)) $\bar K$ is separable and $F$ is just a subset of it.

\problem{V.3.12}
\subpart{(a)}
Let $f\in K[x]$ and $g\in E[x]$ be the minimal polynomials of $u\in F$. Then $g|f$ because $f$ is also in $E$ just that not necessarily minimal. We know that all the roots of $f$ are simple, thus $f= (x-u_i)^1h$ with $h(u_i)\neq 0$. Since $g$ divides $f$, $u_i$ is also a simple root of $g$.
\subpart{(b)}
By part (a), every $u\in F$ that is separable over $K$ is also separable over $E$. 
$E$ is a separable extension of $K$ because $E$ is a subfield of $F$.

\problem{V.3.13}
We will show that each of the propositions is equivalent to its respective part in  Theorem V.3.11.

If $F$ is Galois and of finite dimensional over $K$ then it is also algebraic over $K$. 
This implies that $F$ is separable over $K$ and  the splitting field of a  set of polynomials $S$. 
Each $h\in S$ is the irreducible polynomial of an element of a generating set of $F$. 
Again, being $[F:K]=n$ and using Exercise V.3.1 we can further specify $F$ to be the splitting field of just one polynomial, to wit $f=\prod_{h\in S}h$.

Next, the implication (ii)$\implies$(iii) can be used unchanged only observing that $T$ is defined as the collection of all irreducible factors of $f=\prod_{h\in S}h$. 
Finally, this implies $F$ is Galois over $K$ and finite dimensional implies that it is also algebraic.

\problem{V.3.14}

