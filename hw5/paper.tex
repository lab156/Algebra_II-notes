\problem{V.2.6}
\subpart{(a)}
We follow the hint in the book to show that $x$ is algebraic over $K$. 
We begin defining the polynomial $\phi(y)$ in $K(f/g)[y]$:
$$\phi(y) = (f/g)g(y) - f(y)$$
Observe that $\phi(x) =0$ since $g(x)=g$ and $(f/g)g=f$. 
Also, the degree of $\phi$ is always $\max\{ \deg f, \deg g\}$. 
Only the case in which $\deg f= \deg g$ needs explanation. 
In this case, the principal term has the form $(f/g) a\, y^m - b\, y^m$ where $a,b\in K$ and $m$ is the degree of both $f$ and $g$.
The resulting coefficient will be different from 0 because $f/g$ is non-constant (since $f/g\notin K$); while $b$ definitely is.

Next we show that $f/g$ is transcendental over $K$.
If it wasn't, there would exist an irreducible polynomial $K(x) \ni p(f/g)=0$ with $\deg p = n$.
And this would imply $g^np(f/g)=0$, but this product simplifies into a polynomial in the variable $x$.
This is a contradiction to the transcendence of $x$.

To check that $\phi(y)=z\,g(y)-f(y)$ is irreducible, notice that as a polynomial in $K(y)[z]$, $\phi(y)$ is linear. 
Due to $f$ and $g$ being coprime, $C(\phi)=1$ ($\phi$ has no content); thus we can say that  $\phi(y)$ is also irreducible in $K[y][z]=K[z][y]$.
Thus, it is irreducible in $K(z)[y]$.

The above implies, that $\phi$ is in fact the minimal polynomial; thus $$[K(x):K(f/g)]= \max\{ \deg f, \deg g\}$$.

\subpart{(b)}
Let $p,q\in E\backslash K$ then noticing that:
$$[K(x):K(p,q)] \leq \min\{ [K(x):K(p)], [K(x):K(q)]\}$$
This is because as we already know,  $x$ is algebraic for both $K(p)$ and $K(q)$; 
thus the degree of the minimal polynomial in the extension $K(p,q)$ is bounded by the both of the degree of the fields $K(p),K(q)$ separately.  

Now we contruct the extension $E$ by adding all the $f/g\in K(x)$ we may want it to contain; and the inequality persists:
$$[K(x):E] = \min_{f/g\in E} \{[K(x):K(f/g)]\}$$
This is always finite as long as $E\backslash K\neq \emptyset$

\subpart{(c)}
First we check there is an homomorphism from $\sigma \from K[x]\to K(x)$.
Take the evaluation map from the integral domain $K[x]$ to the commutative ring $K(x)$.
Then -- by theorem III.5.5 -- there exists a homomorphism (which is no other than the evaluation map).
Next, notice $\sigma(\psi(x))\neq 0$ if $K[x]\ni \psi(x) \neq 0$. Which means that every nonzero element is mapped to a unit.
Also, it was established in Homework 4 that $S^{-1}K[x] = K(x)$ where $S$ is the set of nonzero polynomials.
Thus we can extend $\sigma$'s domain to be the quotient field $K(x)$ and:
\begin{equation} \label{ecua.1}
\sigma\left(\frac{\phi(x)}{\psi(x)}\right) = \frac{\phi(f/g)}{\psi(f/g)}
\end{equation}
For all $\psi\neq 0$. 
Note that the requiring that $f$ and $g$ are relatively prime comes into play here, since it guarantees that $x$ is mapped into a nonconstant polynomial (which could be a root of $\psi$).

Secondly we characterize  $\aut_K K(x)$.
Let $\sigma \from K(x) \to K(x)$ be an automorphism defined like in equation (\ref{ecua.1}).
Thus, applying $\sigma$ is just replacing every $x$ for $f/g$.
This implies that the image of $\sigma$ is actually contained in $K(f/g)$. 
Then, $\text{image }\sigma \subset K(f/g) \subset K(x)$; moreover, $\sigma$ is onto over $K(x)$.
Then $K(f/g)=K(x)$, or equivalently $[K(x):K(f/g)]=1$ and by Part (a), $\max\{ \deg f, \deg g\}=1$.

On the other hand, assume now that $\max\{ \deg f, \deg g\}=1$, then:
$$f/g= \frac{ax+b}{cx+d}$$
Also, $f,g$ being relatively prime implies that $a\,d- b\,c\neq 0$. 
It is well known that the inverse of these functions has the form:
$$\tau(x) = \frac{dx+-b}{-cx+a}$$
The only thing we have to check is that $\tau$ is a well-defined homomorphism.
This is the case, since  $\tau$ has the same non-zero ``determinant'' as $\sigma$.
And it was shown earlier that this, in addition to being of max degree 1, implies that $\tau$ is a homomorphism.

\subpart{(d)}
The sets are equal because of the characterization shown above. 
This says that $\sigma\in \aut_K K(x)$ iff $\max\{ \deg f, \deg g\}=1$ (with $f,g$ relatively prime).
At the same time this is equivalent to the rational function having the form:
$$\sigma(x) = \frac{ax+b}{cx+d}$$
With $a\,d- b\,c\neq 0$. 
 
