\textbf{Algebra II Assignment 2 \hspace{\fill} Luis Berlioz}
\subsection*{Problem IV.6.1}
First we prove that all ideals of $R$ are principal. Then that it is an integral domain. 

Let $I$ be any ideal of $R$. $I$ can be viewed as an $R$-submodule of $R$. 
Of course, $R$ is a free $R$-module, then $I$ is also free.
Note that a basis of $I$ can have at most one element by the \emph{hint} given in the problem. 
Also, it cannot be empty. Leaving a basis of one element as the only option; i.e. $I$ is principal.

To check that $R$ is an Ideal Domain. 
For any $p\in R$, if $p\neq 0$, the set $Rp={rp\from \forall r\in R}$ is a non-zero ideal of $R$. 
So by the same argument as above it is also a free submodule. Additionally, it 
has a basis of one element (for example $\{p \}$). 
Then, $Rp$ and $R$  are $R-module$ isomorphic (not necessarily ring isomorphic) and this implies $rp=0$ implies $r=0$.

\subsection*{Problem IV.6.2}
Let $R$ be the integral domain with identity and, $F$ be any free $R$-module. 
Then by theorem IV.2.1 there is an $R$-module isomorphism between $F$ and $\bigoplus R$. 

Now, for the sake of contradiction, assume that $F$ is \emph{not torsion free}. This implies that there exists $r\in R$ and $a\in f$ both non-zero; such that $ra=0$.
 Since $a\neq 0$ corresponds through the isomorphism to a tuple with at least one non-zero element of $R$; which we will call $s$. 
This forces a contradiction, because on one hand we're saying that $R$ is an \emph{integral domain} and on the other, that two non-zero elements, namely $r\,s =0$. 

As a counterexample of the converse,  $\QQ$ seen as a $\ZZ$-module does the job. 
It is clearly torsion free, yet it's not a \emph{free} $\ZZ$-module.
To check the last assertion, note any two elements in $\QQ$ are linearly dependent. 
Nonetheless, these two \emph{abelian groups} are not isomorphic.

\subsection*{Problem IV.6.3}
\subsubsection*{Part (a)}
Since $s$ is \emph{relatively prime} to $r$, then, and using that $R$ is a PID and theorem 3.11; any $p\in R$ can be written as $p=p_1 s+ p_2 r$. 

If we call $a$ the generator of $A$ then any $b\in A$ can be written as $pa=b$. If additionally $p=p_1 s+ p_2 r$; then we see that $b=(p_1s+ p_2 r)a=p_1sa$. The last one clearly being in $sA$. The other inclusion: $sA\subset A$ is by definition.

To prove that $A[s]=0$, take any $\alpha \in A[s]$, then, since $\alpha \in A$ then it can be written as $\alpha=ka$ for $k\in R$ and $a\in A$:
$$s\alpha=0=ska$$
Thus, $sk\in (r)$, which means $k=0$ since $s$ and $k$ are relatively prime.

\subsubsection*{Part (b)}
Since $A=Ra$ then any element in $sA$ can be written as $sua$ with $u\in R$. We will show that $\phi \from sA \to R/(k)$ given by:
$$\phi(sua)= u+(k)$$
is any isomorphism.

First we prove $\phi$ is well-defined. Take $u,v\in R$ such that
$$sua = sva$$
Then it checks that $s(u-v)a= 0$ or equivalently, $(u-v)\in ( r)$.
The rest of the properties are quite simple:
\begin{gather*}
\phi(s(u+v)a)= \phi(sua) + \phi(sva)= u+v+(k)\\
\phi(s(vu)a)= v\phi(sua)= vu+(k)
\end{gather*}
The inverse is $\psi(u+(k)) = sua$. $\psi$ is well-defined since, if $u$ and $v$ are equivalent, that is $v=u+\xi r$ then the last summand always cancels out because of the multiple of $r$; namely: $s\xi r a=0$. The rest of the properties are just consequence of the definition of the module.

To check that $\psi$ and $\phi$ are inverses, note that:
$$\psi(\phi(sua))= \psi(u+(k))= sva$$
where $u$ and $v$ may be different elements of the same coset $u+(r)$ but the whole product as elements of $A$ are equal i.e. $sua = sva$. Similarly, 
$$\phi(\psi(u+(k))= \phi(sua) =v+(k)$$.

Finally, note that $kA=A[s]$. This is because any element in $kA$, say $kua$, is ``killed'' when multiplied by $s$. Conversely, if $\alpha\in A[s]$ then $s\alpha=0$ if we write $\alpha = ua$ then $su\in (r)$. Which implies $k|u$. Then it checks that: 
$$A[s] = kA = R/(s)$$

\subsection*{Problem 4}
\subsubsection*{Part (i)}
Take any submodule $B\subset Ra$ then $B=\{sa \from s\in R\}$. The set of these coefficients is an ideal in $R$ namely:
$$(s_0) = \{s\from sa \in B\}$$
And this means that $B$ is generated by the element $s_0 a$. 

$s_0a$ being an element of $Ra$ generates the order ideal $\o_{sa}=\{k\from ks_0a =0\}$. Call $(k) = \o_{sa}$ and noting that $r\in (k)$, we get that $k|r$.

\subsubsection*{Part (ii)}
If $(r)\subset (s)$, then we have $s|r$. If we call $ts=r$; then we using problem 3 above:
$$Rta =\frac{R}{(s)}$$


\subsection*{Problem 6}
Call $g=\gcd(r,s)$ and $\ell = lcd(r,s)$, $as=\ell$ and $br=\ell$.  Then $g\ell = rs$, we have the following homomorphism $\phi\from R/(s)\oplus R/(t) \to R/(\ell)$:
$$(x+(s), y+(t)) \mapsto ax+by+(\ell)$$
which is surjective but not 1-to-1. By the uniqueness statement in theorem IV.6.12; and noticing that $g$ is also an invariant factor, let's us conclude that $R/(s)$ is the missing piece (or maybe not). 
\subsection*{Problem 7}
Snow
