\subsection*{Problem IV.6.1}
First we prove that all ideals of $R$ are principal. Then that it is an integral domain. 

Let $I$ be any ideal of $R$. $I$ can be viewed as an $R$-submodule of $R$. 
Of course, $R$ is a free $R$-module, then $I$ is also free.
Note that a basis of $I$ can have at most one element by the \emph{hint} given in the problem. 
Also, it cannot be empty. Leaving a basis of one element as the only option; i.e. $I$ is principal.

To check that $R$ is an Ideal Domain. 
For any $p\in R$, if $p\neq 0$, the set $Rp={rp\from \forall r\in R}$ is a non-zero ideal of $R$. 
So by the same argument as above it is also a free submodule. Additionally, it 
has a basis of one element (for example $\{p \}$). 
Then, $Rp$ and $R$  are $R-module$ isomorphic (not necessaryly ring isomorphic) and this implies $rp=0$ implies $r=0$.

\subsection*{Problem IV.6.2}
Let $R$ be the integral domain with identity and, $F$ be any free $R$-module. 
Then by theorem IV.2.1 there is an $R$-module isomorphism between $F$ and $\bigoplus R$. 

Now, for the sake of contradiction, assume that $F$ is \emph{not torsion free}. This implies that there exists $r\in R$ and $a\in f$ both non-zero; such that $ra=0$.
 Since $a\neq 0$ corresponds through the isomorphim to a tuple with at least one non-zero element of $R$; which we will call $s$. 
This forces a contradiction, because on one hand we're saying that $R$ is an \emph{integral domain} and on the other, that two non-zero elements, namely $r\,s =0$. 

As a counterexample of the converse,  $\QQ$ seen as a $\ZZ$-module does the job. 
It is clearly torsion free, yet it's not a \emph{free} $\ZZ$-module.
To check the last assertion, note any two elements in $\QQ$ are linearly dependent. 
Nontheless, these two \emph{abelian groups} are not isomorphic.
