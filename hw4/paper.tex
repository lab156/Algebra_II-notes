\problem{V.1.19}
First we review that at most $mn$ terms will be needed to generate $K(u,v)$.
Any element in $a_j\in K(u)$ is of the form:
$$a_j = \sum_{i=0}^{m-1} b_{ij} u^i$$
Where $m=[K(u):K]$ and $b_{ij}\in K$. 
Additionally, each $c\in K(u,v)$ can be written in the form:
$$c= \sum_{j=0}^{n-1} a_{j} v^j$$
This, because $K(u,v)= K(u)(v)$. By expanding $a_j$ we get that any $c\in K(u,v)$:
$$c= \sum_{j=0}^{n-1} \sum_{i=0}^{m-1} b_{ij} u^i v^j$$
This means that no more than these terms are needed to generate the field $K(u,v)$.

We will prove the second part by means of the counterpositive: 
$$[K(u,v):K]<mn \implies (m,n)>1$$ 
Note that the following couple of equations hold:
\begin{gather*}
m[K(u,v):K(u)] = [K(u,v):K]\\
n[K(u,v):K(v)] = [K(u,v):K]
\end{gather*}
Then:
$$[K(u,v):K] = \frac{m\, n}{\gcd(m,n)}p$$
Our original assumption was that $[K(u,v):K]<mn$ so we conclude $\gcd(m,n)>1$. 

\problem{V.1.20}
\subsection*{Part (a)}
Since:
\begin{gather} \label{eq.1}
    [ML:M][M:K] = [LM:K]\\
    [ML:L][L:K] = [LM:K]
\end{gather}
And by Theorem V.1.2 we see that $[LM:K]$ is finite iff both $[ML:L]$ and $[L:K]$.
The same applies to $[ML:M]$ and $[M:K]$. 

\subsection*{Part (b)}
Using equations $\ref{eq.1}$ we get:
    $$[LM: K]^2= [LM:L][LM:M] \ [M:K][L:K]$$
    Since both $[LM:L]$ and $[LM:M]$ are greater that 1 and $[LM:K]$ has to be a pos; we get the result:
\subsection*{Part (c)}
\subsection*{Part (d)}
\problem{V.1.21}
