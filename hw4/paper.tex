\problem{V.1.19}
First we review that at most $mn$ terms will be needed to generate $K(u,v)$.
Any element in $a_j\in K(u)$ is of the form:
$$a_j = \sum_{i=0}^{m-1} b_{ij} u^i$$
Where $m=[K(u):K]$ and $b_{ij}\in K$. 
Additionally, each $c\in K(u,v)$ can be written in the form:
$$c= \sum_{j=0}^{n-1} a_{j} v^j$$
This, because $K(u,v)= K(u)(v)$. By expanding $a_j$ we get that any $c\in K(u,v)$:
$$c= \sum_{j=0}^{n-1} \sum_{i=0}^{m-1} b_{ij} u^i v^j$$
This means that no more than these terms are needed to generate the field $K(u,v)$.

We will prove the second part by means of the counterpositive: 
$$[K(u,v):K]<mn \implies (m,n)>1$$ 
Note that the following couple of equations hold:
\begin{gather*}
m[K(u,v):K(u)] = [K(u,v):K]\\
n[K(u,v):K(v)] = [K(u,v):K]
\end{gather*}
Then:
$$[K(u,v):K] = \frac{m\, n}{\gcd(m,n)}p$$
Our original assumption was that $[K(u,v):K]<mn$ so we conclude $\gcd(m,n)>1$. 

\problem{V.1.20}
\subsection*{Part (a)}
Since:
\begin{gather} 
    [ML:M][M:K] = [LM:K]\label{eq.1}\\
    [ML:L][L:K] = [LM:K]\label{eq.2}
\end{gather}
And by Theorem V.1.2 we see that $[LM:K]$ is finite if and only if both $[ML:L]$ and $[L:K]$ are.
The same applies to $[ML:M]$ and $[M:K]$. 

\subsection*{Part (b)}
Using equations (\ref{eq.1}) and (\ref{eq.2}) we get that $[M:K]| [LM:K]$ and $[L:K]| [LM:K]$. 

To check the second proposition, i.e. that $[LM:K]\leq [L:K][M:K]$. Note that if $\ell \in LM = L(M)$ then the dimension of $LM$ over $L$ is at most that of $LM$ over $K$, thus:
$$\ell = \sum_{k=1}^{[M:K]}\alpha_k m_k$$
Where the  $m_k\in M$ and the $\alpha_k \in L$ which have dimension $[L:K]$ over $K$ and thus the whole thing can be expressed as:
$$\ell =  \sum_{k=1}^{[M:K]}\sum_{j=1}^{[L:K]}\kappa_{j,k}m_kn_j$$
Where the $\kappa_{j,k}\in K$ and the $n_j\in L$. 
Note that although we may need to use different elements $m_kn_j$ the upper limit of both sums will always remain the same (in total at most $[M:K][L:K]$ terms). 
Therefore, the dimension of $[LM:K]$ is at most $[L:K][M:K]$. 
\subsection*{Part (c)}
The proof goes the same way a in Problem V.1.19 above.
In which we can take $L=K(u)$ and $M=K(v)$ because the hypothesis of the fields being finitely generated is unnecessary.
And it was not used in the proof above.

\subsection*{Part (d)}
If $L$ es algebraic over $K$ then $[L:K]$ is finite. 
Same with $[M:K]$, then:
$$[LM:K]\leq [L:K][M:K]$$
Thus, it is finite, which implies that any element in $LM$ has finite order.
Therefore it is finite.

\problem{V.1.21}
\subsection*{Part (a)}
We will prove this proposition by contradiction. Assume that $\ell \in L\cap M\backslash K$ 

