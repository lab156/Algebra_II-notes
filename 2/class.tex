\subsection{Tensor Products}
\begin{examples}
Let $V$ be a vector space on $\RR$ the inner product $\langle\ ,\ \rangle$ is a bilinear function.
\end{examples}

Suppose $A,B$ are left and right modules respectively, and $f\from A\times B \to C$ where $C$ is an abelian group. We want the \emph{Universal Linear Map} so for any bilinear $f$ the diagram commutes:
\[\xymatrix{
A\times B\ar[r]^g\ar[dr]_f     &      D\ar[d]^h    \\
  & C     } \]
The set where all the pairs $(a,b)$ end up is the tensor product:
\begin{ddef}[Tensor Product]
Let $A$ be a right module and $B$ a left module over a ring $R$. Let $F$ be the free abelian group on the set $A\times B$. Let $K$ be the subgroup of $F$ generated by the following equivalence relations:
\begin{enumerate}[(i)]
\item $(a+a',b)\sim (a,b) + (a',b) $
\item idem for the second component.
\item $(a\,r,b) \sim (a,r\,b)$
\end{enumerate}
The quotient group $F/K$ is called the tensor product. $\imath\from A\times B \to A\otimes_R B $ is called the \textbf{Canonical  Middle Linear Map}
\end{ddef}
\begin{teorema}
The diagram above becomes:
\[\xymatrix{
A\times B\ar[r]^\imath\ar[dr]_{f}     &      A\otimes B\ar[d]^{\exists ! g}    \\
  & C     } \]
\begin{proof}

\end{proof}
\end{teorema}


