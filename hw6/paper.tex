\problem{V.3.9}
Assuming that $F$ is the algebraic closure of $K$, let $E$ be any algebraic field extension.
Let us call $\bar E$ the algebraic closure of $E$, which we know to exist by Theorem V.3.6.
Also by  Theorem V.3.6, we know that $\bar E \cong F$, and say $\sigma$ is the $K-$isomorphism between these two fields.
Therefore the restriction $\sigma|_E$ is $K-$monomorphism that we want

Conversely, let $f\in K[x]$. We will show that $f$ has a root in $F$.
Let $E$ be an extension over $F$ such that $f$ splits in $E$; and $u\in E$ a root of $f$.
We are assuming that there exists a $K-$monomorphism $\sigma\from E\to F$.
Therefore $\sigma(f(u))=f(\sigma(u))=0$, and $\sigma(u)\in F$ is a root of $f$. 

\problem{V.3.11}
\subsection*{both part (a) and (b)}
Let $T\subset K[x]$ be the set of all the irreducible polynomials of all $u\in F$. That is:
$$T=\{ f\in K[x] \from f \text{ is the irreducible polynomial of some } u\in F\}$$
Also, take $\bar K$ to be a splitting field of $T$ over $K$. 

There is no point in making a distinction between the finite and the infinitely generated cases. This is because we would unavoidably use the same method in proving both scenarios; just that in different levels of generality.

Now we can assert that $\bar K$ is a splitting field of a set of separable polynomials $T$.
Thus by theorem V.3.11 ((iii)$\implies $(ii)) $\bar K$ is separable and $F$ is just a subset of it.

\problem{V.3.12}
\subpart{(a)}
Let $f\in K[x]$ and $g\in E[x]$ be the minimal polynomials of $u\in F$. Then $g|f$ because $f$ is also in $E[x]$ just that not necessarily minimal. We know that all the roots of $f$ are simple, thus $f= (x-u_i)^1h$ with $h(u_i)\neq 0$. Since $g$ divides $f$; if $g(u_i)=0$ then it is also has to be a simple root of $g$.
\subpart{(b)}
By part (a), every $u\in F$ that is separable over $K$ is also separable over $E$. 
$E$ is a separable extension of $K$ because $E$ is a subfield of $F$.

\problem{V.3.13}
We will show that each of the propositions is equivalent to its respective part in  Theorem V.3.11.

If $F$ is Galois and of finite dimensional over $K$ then it is also algebraic over $K$. 
This implies that $F$ is separable over $K$ and  the splitting field of a  set of polynomials $S$. 
Each $h\in S$ is the irreducible polynomial of an element of a generating set of $F$. 
Again, being $[F:K]=n$ and using Exercise V.3.1 we can further specify $F$ to be the splitting field of just one polynomial, to wit $f=\prod_{h\in S}h$.

Next, the implication (ii)$\implies$(iii) can be used unchanged only observing that $T$ is defined as the collection of all irreducible factors of $f=\prod_{h\in S}h$. 
Finally, this implies $F$ is Galois over $K$ and finite dimensional implies that it is also algebraic.

\problem{V.3.14}
First we briefly prove part (a) of problem V.1.5: If $K\subset L\cap M$ and $L=K(S)$ for some $S\subset L$, then $LM = M(S)$.

For any $\ell \in L=K(S)$ it can be written as a liner combination $\ell=\sum k_i\alpha_i$. Similarly, every $u\in ML=M(L)$ is of the form:
\begin{align*}
u &= \sum m_i \ell_i\\
  &= \sum m_i \sum k_j \alpha_j\\
  &= \sum_{i,j} m_i k_j \alpha_j
\end{align*}
The coefficients $m_ik_j\in M$ so we have written any $u\in ML$ as a linear combination of elements in $S$.

Since we are told $L$ is Galois and finite dimensional over $K$, we know by problem V.3.13 above that $L$ is the splitting field over $K$ of a polynomial $f\in K[x]$ whose irreducible factors are separable.
Let $S$ be the set of all the roots of $f$ that spawn $L$, then $L= K(S)$ and thus $ML=M(S)$ by the result above.
Observe this implies that $ML$ is Galois over $M$ because $f$ also splits over $ML$ and every element in $S$ is still separable over the intermediate field $M$ by problem V.3.12.

Finally, we put in evidence the group isomorphism: $\aut_{L\cap M} K(S) \cong \aut_MM(S)$. 
For all $\sigma \in \aut_{L\cap M} K(S)$ we have that $\sigma$ is completely determined by its action on each of the $\alpha_i \in S$; with the caveat that if $\alpha_i \in M$ then it is \emph{left unchanged} i.e. for $k_1\ldots k_n \in K$:
$$\sigma(k_1 \alpha_1+\ldots + k_n \alpha_n) = k_1 \sigma(\alpha_1)+\ldots +  k_n \sigma(\alpha_n)$$
Using this characterization, we take the isomorphism to be the map 
$$\Psi \from \aut_{L\cap M} K(S) \to \aut_MM(S)$$
Such that  $\Psi\sigma$ has the same action on $S$ as $\sigma$. To be precise:
$$\sigma(\alpha_i)=(\Psi\sigma)\alpha_i$$
And fixes all the elements of $M$ (including the ones in $S$). Observe that $\Psi\sigma\in \aut_MM(S)$ because for any $m_1\ldots m_n \in M$:
$$\Psi\sigma(m_1 \alpha_1+\ldots + m_n \alpha_n) = m_1 \Psi\sigma(\alpha_1)+\ldots +  m_n \Psi\sigma(\alpha_n)$$
It is a well-defined map since it unambiguously acts on the roots $\alpha_i$ leaving $M$ fixed; and again this \#CompletelyDeterminesAnAutomorphism.
To show that $\Psi$ is a group homomorphism, take $\sigma, \tau\in   \aut_{L\cap M} K(S)$. We need to show that for all $\alpha_i \in S$:
$$\Psi(\sigma \tau)\alpha_i= (\Psi\sigma)(\Psi\sigma)\alpha_i$$   
This is the case since $\Psi(\sigma \tau)\alpha_i =(\sigma\tau)\alpha_i$ by definition and this same definition applied twice implies that 
$$(\Psi\sigma)(\Psi\tau)\alpha_i=(\Psi\sigma)\tau (\alpha_i)=(\sigma\tau)\alpha_i$$
To show that $\Psi$ has trivial kernel, let $\Psi\sigma$ be the identity automorphism. Then $\Psi\sigma(\alpha_i) = \sigma(\alpha_i)=\alpha_i$. Therefore: 
$$\sigma(k_1 \alpha_1+\ldots + k_n \alpha_n) = k_1 \sigma(\alpha_1)+\ldots +  k_n \sigma(\alpha_n)=k_1 \alpha_1+\ldots + k_n \alpha_n$$
In other words, $\sigma$ is the identity.

\problem{V.3.15}
\subpart{(a)}
By Theorem V.3.12 part (ii'), we know that $F$ is Galois over every $E$. To check that $F$ is an algebraic extension of $E$, take any $u\in F$, then $u$ is a root of a polynomial  $f\in K[x]$ thus also  $f\in E[x]$. Therefore $F$ is an algebraic extension of $E$.
\subpart{(b)}
We will show that all $\sigma\in \aut_KE$ are extendible namely to at least one $\bar\sigma \in \aut_K F$. By problem V.2.12 which was proven on homework 5, we get that $F$ is Galois over $K$.

Let $S$ be the set of all roots of the family of polynomials in $K[x]$ that $F$ splits, then $F=E(S)$. 
For any $\sigma\in \aut_KE$, if it is going to extend to $F$ at all then it has to satisfy for any $F\ni u=e_1 r_1 + \ldots e_n r_n,\ e_i\in E$ and $r_i\in S$:
$$\sigma(e_1 r_1 + \ldots e_n r_n)=\sigma(e_1) \sigma(r_1) + \ldots +\sigma(e_n) \sigma(r_n)$$
If we choose $\sigma(r_i)=r_i$ for all $i=1,\ldots,n$, we assert this gives a an automorphism in $\aut_KF$.
