\problem{V.2.7}
\subpart{(a)}
We follow the hint in the book to show that $x$ is algebraic over $K$. 
We begin defining the polynomial in $K(f/g)[y]$:
$$\phi(y) = (f/g)g(y) - f(y)$$
Observe that $\phi(x) =0$ since $g(x)=g$ and $(f/g)g=f$. 
Also, the degree of $\phi$ is always $\max\{ \deg f, \deg g\}$. 
Only the case in which $\deg f= \deg g$ needs explanation. 
In this case, the principal term has the form $(f/g) a\, y^m - b\, y^m$ where $a,b\in K$ and $m$ is the degree of both $f$ and $g$.
The resulting coefficient will be different from 0 because $f/g$ is non-constant (since $f/g\notin K$); while $b$ definitely is.

Next we show that $f/g$ is transcendental over $K$.
If it wasn't, there would exist an irreducible polynomial $K(x) \ni p(f/g)=0$ with $\deg p = n$.
And this would imply $g^np(f/g)=0$, but this product simplifies into a polynomial in $x$.
This is a contradiction to the transcendence of $x$.

To check that $\phi(y)=z\,g(y)-f(y)$ is irreducible, notice that as a polynomial in $K(y)[z]$, $\phi(y)$ is linear. 
Due to $(f,g)=1$, then $\phi(y)$ is also irreducible in $K[y][z]=K[z][y]$.
Thus, it is irreducible in $K(z)[y]$.

The above implies, that $[K(x):K(f/g)]= \max\{ \deg f, \deg g\}$.

\subpart{(b)}

