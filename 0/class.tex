\begin{ddef}[Internal Direct Sum]
    Let $A$ be and additive group, and $N_i$ (normal) subgroups. Say $A$ is the internal direct sum of $N_i$ iff $A=\langle \cup N_i\rangle$ and:
    $$N_k \cap\langle \cup_{k\neq i}N_i\rangle=e $$
    Equivalently, see @@hungerford 62@@
\end{ddef}

\begin{ddef}[The free group $F_S$]
    The free group $F_S$ with \emph{free generating set} $S$ is the group generated by all the words with letter in $S\cup S^{-1}$.
\end{ddef}

\begin{ddef}[Free Abelian Group]
    The free abelian group on a Set $S$ is the finite linear combinations of elements of $S$.
\end{ddef}



\begin{ddef}[Modules]
An $R$-module consists of:
\begin{itemize}
\item $R$ is a ring, 
\item $A$ is an additive group.
\item And, the map $R\times A \to A$ (both additions distribute and $r(s\,a)=(r\,s)a)$.
\end{itemize}
Additionally defines:
\begin{description}
    \item[Unitary Module] if $R$ has identity.
    \item[Vector Space] if $R$ is a division ring.
\end{description}
\end{ddef}

\begin{examples}
    \begin{enumerate}
        \item If $I$ is an ideal, then $R/I$ is an $R$ module.
        \item The \textbf{pullback} along a ring homomorphism $\phi$ is:
            $$r\,x \mapsto \phi(r)x$$
        \item The endomorphisms of a ring $\ndo(R)$ with $f\,r \mapsto f(r)$ (brilliant!). 
    \end{enumerate}
\end{examples}

\begin{ddef}[$R$-module homomorphims]
    $f$ is a group homomorphims:
    $$f(a+a') = f(a)+f(a')$$
    And $f$ respects the module operation:
    $$f(r\,a) = rf(a)$$
\end{ddef}

\begin{ddef}[Submodules]
    If $A$ is an $R$-module then $B\subset A$ is a submodule iff:
    \begin{itemize}
        \item $B$ is an additive subgroup.
        \item $\forall r\in R,\ b\in B,\ r\,b\in B$
    \end{itemize}
    Submodules of vector spaces are called \emph{subspaces}.
\end{ddef}

\begin{ddef}[Exact Sequence]
    Let $f,g$ be module homomorphisms such that:
    $$A\xrightarrow{f} B \xrightarrow{g} C$$
    We say that the homomorphisms $f,g$ are exact iff:
    $$\im f = \ker g$$
    And a \textbf{Short Exact Sequence} has the form:
    $$0\to A\xrightarrow{f} B \xrightarrow{g} C\to 0$$
\end{ddef}

\begin{remarks}[Comparison of the free properties]
    Take function $f\from B\to G$ where $G$ is an abelian group. There exists a unique homomorphism $g$ from the object in question to $G$.
    \begin{center}
        \begin{tabular}{|c|l|l|p{3cm}|}
        \hline
                            & $B$ & Object & $\imath$ \\
        \hline
    Free Group $F_S$ & Arbitrary Set $S$ & $F_S$ & Inclusion map \\
        \hline
    Free Abelian Group & Basis $B$ & $F$ Group & linear combinations coefficients in $\ZZ$. \\
        \hline
        $R$-Modules & && \\ 
            \hline
    \end{tabular}
\end{center}
\end{remarks}
