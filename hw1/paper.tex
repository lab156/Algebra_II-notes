\begin{description}
\item[IV.5.2] Let $A$ and $B$ be abelian groups.
\begin{enumerate}[(a)]
\item First we give a bilinear function from the cartesian product $A\times \ZZ_m$ to the quotient group $A/mA$. Then we invoke theorem IV.5.2 and get a homomorphism between the tensor product $A\otimes \ZZ_m$ and $A/mA$; which we proceed to prove is in a fact an isomorphism.

    Let the function $f\from A\times \ZZ_m\to A/mA$ be defined by $$f(a,k+m\ZZ) = ka+mA$$
    Note $f$ is well defined...

    Also $f$ is bilinear...

    Using theorem IV.5.2 get $\bar f$ which is a homomorphism from $f\from A\otimes Z_m \to A/mA$...
\item By the preceding statement, $A= \ZZ_m$... (prove for m=0)
\item $A\otimes B=(\ZZ^n\oplus\ZZ_{p_1}\oplus \ldots \oplus \ZZ_{p_t})\otimes(\ZZ^n\oplus\ZZ_{q_1}\oplus \ldots \oplus \ZZ_{q_t})$
\end{enumerate}
\item[IV.5.3]
    \begin{enumerate}[(a)]
        \item For any tensor $a\otimes p/q \in A\otimes \QQ$, where $|a|=n$:
            $$a\otimes \frac pq = n\,a\otimes \frac p{n\,q} = 0$$
            So this can be done to any coset $\sum r_i(a_i\otimes p_i/q_i$).
        \item $1\otimes Id $ is an epimorphism we only need to prove $\QQ\otimes \QQ \xrightarrow{f} \QQ \to 0$ is also a monomorphism.
    \end{enumerate}
\item[IV.5.4]
\item[IV.5.7]
\item[IV.5.8]
\item[IV.5.9]
\end{description}
