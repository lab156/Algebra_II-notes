\textbf{Problem IV.6.1}First we prove that all ideals of $R$ are principal. Then that it is an integral domain. 

Let $I$ be any ideal of $R$. $I$ can be viewed as an $R$-submodule of $R$. 
Of course, $R$ is a free $R$-module, then $I$ is also free.
Note that a basis of $I$ can have at most one element by the \emph{hint} given in the problem. 
Also, it cannot be empty. Leaving a basis of one element as the only option; i.e. $I$ is principal.

To check that $R$ is an Ideal Domain. 
For any $p\in R$, if $p\neq 0$, the set $Rp={rp\from \forall r\in R}$ is a non-zero ideal of $R$. 
So by the same argument as above it is also a free submodule. Additionally, it 
has a basis of one element (for example $\{p \}$). 
Then, $Rp$ and $R$  are $R-module$ isomorphic (not necessaryly ring isomorphic) and this implies $rp=0$ implies $r=0$. \\[1em]
